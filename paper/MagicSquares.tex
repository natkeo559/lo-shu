\documentclass{rhumj_new}

%\title[short title (for header)]{full title}
\title[Magic Squares]{An Approach to Computing Magic Squares Using High-Performance Computing and Group Theory}

%\author[short author name (for header)]{author name}
%\affiliation{author affiliation}
%\email{author email}

\author[Keough]{Nathan H. Keough\thanks{Thanks to all who encouraged and supported me.}}
\affiliation{Maryville College}
\email{nathan.keough@my.maryvillecollege.edu}

%additional authors; list in alphabetical order by last name
% \author[Euler]{Leonard Euler\thanks{Thanks for nothing, Bernoulli.}}
% \affiliation{University of Basel}
% \email{oiler@notmail.com}

%please include 2010 mathematics subject classification:
%https://mathscinet.ams.org/msc/msc2010.html
% Computer science {For papers containing software, source code, etc. in a specific mathematical area, see the classification number –04 in that area}
\subjclass{68V99}
%please provide keywords for your article
\keywords{magic squares, computational group theory, rust}

%\abstract{A brief abstract goes here.}

\begin{document}
\abstract{This paper introduces a novel algorithm for computing magic squares, exploiting
  group theory concepts such as permutation representation, group operations, and group actions to
  encode symmetries. By defining the group operation as composition, the set as a subset of the
  group of magic squares in a specific order, we may systematically explore the permutations of the
  group and extrapolate information about the magic squares to generate new magic squares not in
  the originating set. This vastly reduces computation times for enumerating the solutions to magic
  squares, while also encoding the symmetries in a manner that is easier to analyze
  programmatically. Overall, this study reveals the profound connection between magic squares and
  group theory, offering promising avenues for symmetry-driven algorithms and applications in
  combinatorial mathematics.}

\section{Introduction}

\subsection{Something Subsection}

Ideally, the introduction should introduce the background of the problem at hand and motivate why mathematicians (and/or others) are interested in studying it. This should be aimed at the level of undergraduates who are not experts in your area. We'd also like the introduction to contain a road map for the article; i.e., a description of what to expect in each section.


\section{The title of the second section}
The class file contains definitions for several environments (e.g., {\ttfamily thm, corollary, lemma}, etc). Examples follow below.

Important definitions should be set in the \texttt{defin} environment.
\begin{defin} \label{prim-def}
Let $p \in \mathbf{N}$ and suppose the following hold:
\begin{itemize}
\item[\sffamily\bfseries P1.] If $d \in \mathbf{N}$ such that $d \mid p$, then $d = 1$ or $d=p$.
\item[\sffamily\bfseries P2.] $p \neq 1$.
\end{itemize}
Then we say that $p$ is a \emph{prime number}.
\end{defin}

Use the \texttt{label} command for reference to these environments later on in your document. The class file uses the \texttt{cleveref} package which determines the type of label being referenced:
\begin{center}
\begin{tabular}{lcr}
 & \LaTeX\ code & Result \\
 \hline
Using \texttt{ref}: & \verb|\textbf{definition \ref{prim-def}}| & \textbf{definition \ref{prim-def}} \\
Using \texttt{cref}: & \verb|\cref{prim-def}| & \cref{prim-def}
\end{tabular}
\end{center}

%an example of the lemma environment
\begin{lemma} \label{baby-ftoa}
If $n \in \mathbf{N}$, then there exists a prime number $p$ such that $p \mid n$.
\end{lemma}

\begin{proof}
Suppose, for contradiction, that there exists $n \in \mathbb{N}$ such that $n >1$ and $n$ is not divisible by any prime. Let
\[ S = \{ n \in \mathbf{N}: n>1,\ \text{$n$ is not divisible by a prime}\}.\]
Then $S$ is a non-empty subset of $\mathbf{N}$. The well-ordering principle\footnote{An excellent principle.} fashions a least element of $S$, say $m$. Note that $m$ is not prime (otherwise it would be divisible by a prime, namely itself). Since $m$ is not prime, there exists $a \in \mathbf{N}$ such that $a \mid m$ and $a \neq 1$ and $a \neq m$. Let $b \in \mathbf{Z}$ such that $ab = m$. Note that $b \in \mathbf{N}$ since $a,m \in \mathbf{N}$. What's more, $b >1$ since otherwise $a=m$. So it must be that $1 < a,b < m$. Since divisibility is transitive, it follows that neither $a$ nor $b$ is divisible by a prime. But then $a \in S$ (and so is $b$) and $a  = m/b < m$ contradicting the minimality of $m$. Thus it must be that $S$ is empty which proves the lemma.
\end{proof}

%an example of the thm environment
\begin{thm}[Euclid] There are infinitely many prime numbers.
\end{thm} \label{inf-prim}

\begin{proof}
Let $p_1,p_2,\ldots, p_k$ be primes, and let 
\[ n = p_1p_2 \cdots p_k + 1.\]
By \cref{baby-ftoa}, there exists a prime $p$ such that $p \mid n$. Suppose, for contradiction, that $p = p_i$ for some $i=1,2,\ldots, k$. It follows that $p \mid n-p_1p_2\cdots p_k = 1$, but this is absurd. So it must be that $p \neq p_i$ for any $i = 1,2,\ldots k$. This shows that a new prime can be constructed from any finite list of primes. Hence the number of primes is not finite. 
\end{proof}

Use the \texttt{cite} command to cite references. For example, the above proof can be found in Euclid's Elements \cite{elements}. See below for how to add citations to the \texttt{thebibliography} environment.

\section{A title for the third section}

There is an \texttt{example} environment.

\begin{exa} \label{geo-series} The geometric series
\[ \sum_{k=0}^{\infty} \frac{1}{2^k} \]
converges to $2$.
\end{exa}

When starting a sentence with a reference, use the \texttt{Cref} command. \Cref{geo-series} can be used to prove that $2$ is not the only prime.

\begin{thm} \label{more-primes} The set of primes includes more numbers than just $2$.
\end{thm}

\begin{proof}
If $2$ were the only prime number, then the fundamental theorem of arithmetic would give us that
\[ \sum_{n=1}^{\infty} \frac{1}{n} = \sum_{k=0}^{\infty} \frac{1}{2^k}.\]
The series on the left diverges whereas the series on the right converges. This is impossible, so there must be more prime numbers beyond the number $2$.
\end{proof}

There is a \texttt{remark} environment as well.

\begin{remark} The idea in \cref{more-primes} can be extended to show that the set of primes includes more numbers than just $2$ and $3$. In fact, Euler went on to prove that if the number of primes were finite, then the harmonic series would converge.
\end{remark}

\section{A title for the fourth section}

There are \texttt{prop} and \texttt{corollary} environments as well.

\begin{prop} \label{sqr-decomp} Every natural number can be written uniquely as a product of a square-free number and a square.
\end{prop}

As a corollary to \cref{sqr-decomp} we obtain

\begin{corollary} \label{low-bound} Let $\pi(n)$ denote the number of primes less than or equal to $n$. Then
\[ \pi(n) \geq \frac{\log n}{2\log 2}.\]
\end{corollary}

\begin{proof} There are no more than $2^{\pi(n)}$ square free numbers less than $n$. Also, there are no more than $\sqrt{n}$ squares less than $n$. It follows from \cref{sqr-decomp} that 
\[ n \leq 2^{\pi(n)} \sqrt{n}.\]
The corollary follows by applying log.
\end{proof}

For an unnumbered remark, use the \texttt{xrem} environment.
\begin{xrem} Note that \cref{low-bound} implies that there are infinitely many primes.
\end{xrem}

\begin{thebibliography}{10}

%items in the bibliography should be listed in alphabetical order
%in order to ensure that your reference is formatted correctly, do the following:
%   1) Visit mathscinet to find your citation; for example, search author=euclid, title=elements. Click on the citation
%   2) Copy (ctrl-c) the reference from mathscinet; for example:
%Euclid
%Euclid's Elements.
%All thirteen books complete in one volume. The Thomas L. Heath %translation. Edited by Dana Densmore. Green Lion Press, Santa Fe, %NM, 2002. xxx+499 pp. ISBN: 1-888009-18-7; 1-888009-19-5
%01A75 (01A20 01A60 51-03)
%   3) paste (ctrl-v) the reference into mref %https://mathscinet.ams.org/mref
%   4) click search, then click TeX
%   5) copy/paste the TeX code from mref as shown below:

\bibitem{elements}
Euclid, {\it Euclid's {\it Elements}}, the Thomas L. Heath translation, Green Lion Press, Santa Fe, NM, 2002. MR1932864

\bibitem{diff-calc}
Euler, {\it Foundations of differential calculus}, translated from the Latin by John D. Blanton, Springer-Verlag, New York, 2000. MR1753095

\bibitem{eulogy}
P. G. L. Dirichlet, Ged\"{a}chtni\ss rede auf Carl Gustav Jacob Jacobi, in {\it Nachrufe auf Berliner Mathematiker des 19. Jahrhunderts}, 6--34, Teubner-Arch. Math., 10, Teubner, Leipzig. MR1104895

\end{thebibliography}

\end{document} 